% HERE: we just present:
% 1. What the SW does (not how)
% 2. What guarntees the user has that it works (tests), incl. CI
% 3. The fact that it is easy to use
% 4. Convince the reader that the SW is professionally made (e.g., good docs) 
%
% Requirements (what do we want the SW to do)
% SW structure (Python code, classes, CUDA, etc)
%   L  Mini code snippets (not CODE!), e.g., via diagrams
%   L  Link to GH repo
% GitHub Actions (explain what tests)
%   L  Code coverage (?)
%   L  WHAT is being tested? 
%   L  Unit test: boundary/initial conditions
% Other information:
%   L  Licence


\section{Purpose of the Software}
\label{sec:software_purpose}
The software developed in this project provides a Python package of numerical solvers for solving partial differential equations, with a focus on the Black-Scholes equation (BSE).
The current state of the software supports the implementation of finite difference methods, including the explicit and Crank-Nicolson schemes, as well as Monte Carlo simulations for option pricing.
To enhance computational efficiency, the software has been extended to leverage GPU acceleration via CUDA, optimising performance for computationally intensive tasks. 

\section{Software Structure} \label{sec:software_structure}

\subsection{High-Level Architecture}
The software is structured into two main libraries: the Python-based solvers and the GPU-accelerated CUDA C++ solvers. Each component is designed to implement the numerical methods discussed in Section \ref{sec:theory_and_background}. These include:
\begin{itemize}
    \item \textbf{PDESolvers:} This Python library implements finite-difference methods, particularly \textit{explicit} and \textit{Crank-Nicolson}, for solving partial differential equations. Initially designed for the heat and Black-Scholes equation, it has since been extended to focus on the application of the Black-Scholes model on option pricing, including the implementation of Monte Carlo simulations for numerical approximations and analytical solutions for exact pricing. Additionally, the library integrates real market data using the \textit{yfinance} package, allowing key financial metrics such as \textit{volatility} and \textit{mean} to be calculated from historical time-series data. 
    \item \textbf{GPUSolver:} The GPU-accelerated library is designed to optimise the performance of the numerical methods implemented in the PDESolvers library by leveraging GPU parallelisation. These solvers take advantage of the CUDA architecture to accelerate the computationally-intensive parts of the calculation, resulting in significant speed-ups compared to the CPU-based implementation of the same methods.
\end{itemize}

\begin{landscape}
\begin{figure}[H]
    \centering
\begin{tikzpicture}[scale=0.46, transform shape]

        \begin{umlpackage}[x=-9, y=8]{pdes}
            \umlclass[y=-11]{BlackScholesEquation}{
            - option\_type: OptionType \\
            - S\_max: float \\
            - K: float \\
            - r: float \\
            - sigma: float \\
            - expiry: float \\
            - s\_nodes: int \\
            - t\_nodes: int \\
            - V: np.ndarray 
            }{
                \umlstatic{+ generate\_grid(value: float, nodes: int)} \\
                + s\_nodes: int \{property\} \\
                + t\_nodes: int \{property\} \\
                + option\_type: OptionType \{property\} \\
                + S\_max: float \{property\} \\
                + sigma: float \{property\} \\
                + expiry: float \{property\} \\
                + rate: float \{property\} \\
                + strike\_price: float \{property\}
            }

            \umlclass[y=0]{HeatEquation}{
                - length: float \\
                - x\_nodes: int \\
                - time: float \\
                - t\_nodes: int \\
                - k: float \\
                - initial\_temp: callable \\
                - left\_boundary\_temp: callable \\
                - right\_boundary\_temp: callable
            }{
                + set\_initial\_temp(u0: callable) \\
                + set\_left\_boundary\_temp(left: callable) \\
                + set\_right\_boundary\_temp(right: callable) \\
                + generate\_grid(value: float, nodes: int) \\
                + length: float \{property\} \\
                + time: float \{property\} \\
                + x\_nodes: int \{property\} \\
                + t\_nodes: int \{property\} \\
                + k: float \{property\} \\
                + get\_initial\_temp(x: float): float \\
                + get\_left\_boundary(t: float): float \\
                + get\_right\_boundary(t: float): float
            }
        \end{umlpackage}

        \begin{umlpackage}[x=1, y=12]{solvers}
            \umlclass[x=0, y=0]{BlackScholesExplicitSolver}{
                - equation : BlackScholesEquation
            }{
                + solve()
            }
            
            \umlclass[x=6.5, y=0]{BlackScholesCNSolver}{
                - equation: BlackScholesEquation
            }{
                + solve()
            }

            \umlclass[x=12.3, y=0]{Heat1DExplicitSolver}{
                - equation : HeatEquation
            }{
                + solve()
            }

            \umlclass[x=17.5, y=0]{Heat1DCNSolver}{
                - equation : HeatEquation
            }{
                + solve()
            }

            \umlclass[type=abstract, x=2, y=-4]{Solver}{

            }{
                + solve()
            }

            \umlinherit[geometry=-|]{BlackScholesExplicitSolver}{Solver}
            \umlinherit[geometry=|-]{BlackScholesCNSolver}{Solver}
            \umlinherit[geometry=|-]{Heat1DExplicitSolver}{Solver}
            \umlinherit[geometry=|-]{Heat1DCNSolver}{Solver}

        \end{umlpackage}

        \begin{umlpackage}[x=3, y=1]{solution}
            \umlclass[y=-0.5]{Solution}{
                - result: numpy.array \\
                - x\_grid: numpy.array \\
                - y\_grid: numpy.array \\
                - dx: float \\
                - dy: float \\
                - duration: float 
            }{
                + plot() \\ 
                + get\_result() \\ 
                + get\_execution\_time() \\
                + \_\_sub\_\_(other: Solution) \\
                \# set\_plot\_labels(ax)
            }

            \umlclass[x=6, y=2]{Solution1D}{
                
            }{\# set\_plot\_labels(ax)}

            \umlclass[x=8, y=-3]{SolutionBlackScholes}{
                - option\_type: OptionType \\
                - delta: float \\
                - gamma: float \\
                - theta: float \\
            }{
                + plot\_greek(greek\_type: Greeks, time\_step=int) \\
                \# set\_plot\_labels(ax)
            }

            \umlinherit[geometry=|-]{Solution1D}{Solution}
            \umlinherit[geometry=|-]{SolutionBlackScholes}{Solution}
        \end{umlpackage}

        \begin{umlpackage}[x=-10, y=-12.5]{enums}
            \umlenum{OptionType}{
                EUROPEAN\_CALL \\
                EUROPEAN\_PUT
            }{}

            \umlenum[x=4]{Greeks}{
                DELTA \\
                GAMMA \\
                THETA
            }{}
        \end{umlpackage}

        \begin{umlpackage}[x=26, y=2]{utils}

            \umlclass{Heat1DHelper}{
            }{
                \umlstatic{\# build\_tridiagonal\_matrix(a: float, b: float, c: float, nodes: int)}
            }

            \umlclass[y=-3]{BlackScholesHelper}{
            }{
                \umlstatic{\# calculate\_greeks\_at\_boundary(equation, delta, gamma, theta, tau, V, S, ds)} \\
                \umlstatic{\# set\_boundary\_conditions(equation, T, tau)}
            }

            \umlclass[y=-8]{RBFInterpolator}{
                - z: np.ndarray \\
                - hx: float \\
                - hy: float \\
                - nx: int \\
                - ny: int
            }{
                - get\_coordinates(x: float, y: float) \\
                - euclidean\_distances(x\_minus: float, y\_minus: float, x: float, y: float) \\
                - \umlstatic{rbf(d: float, gamma: float)} \\
                + interpolate(x: float, y: float)
            }

            \umlclass[y=-13.5]{GPUResults}{
                - file\_path: str \\
                - s\_max: float \\
                - expiry: float \\
                - grid\_data: np.ndarray
            }{
                + get\_results()\\
                + plot\_option\_surface()
            }
            
        \end{umlpackage}

        \begin{umlpackage}[x=4, y=-12]{optionspricing}
            \umlclass[x=0,y=-4]{BlackScholesFormula}{
                - option\_type: OptionType \\
                - S0: float \\
                - strike\_price: float \\
                - r: float \\
                - sigma: float \\
                - expiry: float
            }{
                + get\_black\_scholes\_merton\_price()
            }

            \umlclass[x=9,y=0]{MonteCarloPricing}{
                - option\_type: OptionType \\
                - S0: float \\
                - strike\_price: float \\
                - r: float \\
                - sigma: float \\
                - T: float
                - time\_steps: float \\
                - sim: int \\
                - S: np.ndarray \\
                - run: bool
                - duration: float
            }{  
                - generate\_grid() \\
                + simulate\_gbm() \\
                + get\_monte\_carlo\_option\_price() \\
                + get\_execution\_time() \\
                + plot()
            }

            \umlclass[x=0,y=2]{HistoricalStockData}{
                - stock\_data: pd.DataFrame \\
                - ticker: str 
            }{
                + fetch\_stock\_data(start\_date: str, end\_date: str) \\
                + estimate\_metrics() \\
                + get\_latest\_stock\_price() 
            }

        \end{umlpackage}

        \umlcompo{Solver}{Solution}
        \umlassoc[geometry=|-]{MonteCarloPricing}{OptionType}
        \umlassoc{BlackScholesFormula}{OptionType}
        \umlassoc{BlackScholesEquation}{OptionType}
        \umlassoc{SolutionBlackScholes}{OptionType}
        \umlassoc{SolutionBlackScholes}{Greeks}
        \umldep{BlackScholesExplicitSolver}{BlackScholesEquation}
        \umldep{BlackScholesCNSolver}{BlackScholesEquation}
        \umldep{Heat1DExplicitSolver}{HeatEquation}
        \umldep{Heat1DCNSolver}{HeatEquation}
        \umldep{BlackScholesExplicitSolver}{BlackScholesHelper}
        \umldep{BlackScholesCNSolver}{BlackScholesHelper}
        \umldep{Heat1DCNSolver}{Heat1DHelper}

\end{tikzpicture}
\end{figure}
\end{landscape}


\section{Testing and Reliability} \label{sec:software_testing}
The software has undergone comprehensive testing to verify its correctness and functionality. These tests include checks for boundary conditions, initial conditions and performance benchmarks to ensure that the numerical methods produce accurate and reliable results. To streamline the development process, a CI/CD pipeline has been established using GitHub Actions. This automated pipeline runs tests whenever any changes are pushed to the main branch, ensuring that new feature additions or modifications do not introduce bugs or errors to the software. Code coverage is also monitored to ensure that the tests do not miss any critical parts of the code, providing confidence in the reliability of the software.

\subsection{Testing Details}
The testing framework used in this project is \textit{pytest}, which is a widely-used testing framework for Python. The tests are designed to cover various aspects of the software, including:
\begin{itemize}
    \item \textbf{Unit Tests:} These tests validate individual components of the software, ensuring that each function and method behaves as expected. Specifically, they verify that the boundary and initial conditions of the numerical methods are correctly implemented, checking that appropriate constraints are applied throughout the calculations. The tests also include validation for incorrect or unsupported inputs, raising appropriate exceptions to prevent runtime errors.
    \item \textbf{Integration Tests:} Integration tests check the interaction between multiple components of the software to ensure they work correctly. For this instance, the tests focused on comparing the results of different solvers, verifying the approximated solutions against their analytical counterparts, and simulating specific behaviour via mock data. These were particularly done by testing the convergence of the numerical methods from the absolute differences of the approximated solutions or interpolated solutions. Additionally, the Monte Carlo simulations were validated by using mocked data to simulate specific scenarios, ensuring that the simulation produced expected results. 
\end{itemize}
